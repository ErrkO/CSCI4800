\documentclass[12pt]{article}
%\title{%
%    3D Printing technologies \\
%%    \large An Abstract
%}
%\author{Eric Oliver}
%\date{\today}

%\setcounter{chapter}{0}

\usepackage{listings}
\usepackage{times}
\usepackage{mathtools}
\usepackage{color}
\usepackage{graphicx}
\usepackage{csquotes}
\usepackage[margin=1in]{geometry}

\usepackage{setspace}
\doublespacing

\definecolor{codegreen}{rgb}{0,0.6,0}
\definecolor{codegray}{rgb}{0.5,0.5,0.5}
\definecolor{codepurple}{rgb}{0.58,0,0.82}
\definecolor{backcolour}{rgb}{0.95,0.95,0.92}

\lstdefinestyle{mystyle}{
    backgroundcolor=\color{backcolour},   
    commentstyle=\color{codegreen},
    keywordstyle=\color{magenta},
    numberstyle=\tiny\color{codegray},
    stringstyle=\color{codepurple},
    basicstyle=\footnotesize,
    breakatwhitespace=false,         
    breaklines=true,                 
    captionpos=b,                    
    keepspaces=true,                 
    numbers=left,                    
    numbersep=5pt,                  
    showspaces=false,                
    showstringspaces=false,
    showtabs=false,                  
    tabsize=4
}

\lstset{style=mystyle}

\usepackage{fancyhdr}
\pagestyle{fancy}
\lhead{} 
\chead{} 
\rhead{Oliver \thepage} 
\lfoot{} 
\cfoot{} 
\rfoot{} 
\renewcommand{\headrulewidth}{0pt} 
\renewcommand{\footrulewidth}{0pt} 
%To make sure we actually have header 0.5in away from top edge
%12pt is one-sixth of an inch. Subtract this from 0.5in to get headsep value
\setlength\headsep{0.333in}

%
%Works cited environment
%(to start, use \begin{workscited...}, each entry preceded by \bibent)
% - from Ryan Alcock's MLA style file
%
\newcommand{\bibent}{\noindent \hangindent 40pt}
\newenvironment{workscited}{\newpage \begin{center} Works Cited \end{center}}{\newpage }

\begin{document}
\begin{flushleft}

Eric Oliver \\
Dr. Li \\
CSCI 4800-12 \\
\today

\begin{center}
\large 3D Printing Technologies
\end{center}

\setlength{\parindent}{0.5in}

%%%%%%%%%%%%%
%   Body    %
%%%%%%%%%%%%%

% #3dp = 3D print
% #AM = additive manufacturing
% $3d = three dimensional

\#3dping is a relatively new form of additive manufacturing; it is the process by which you add more and more material to create a three dimensional object. There are a few different types of \#3dpers that all do the same thing but at varying accuracies, speeds, and affordability. 3D printing has taken off in the past ten years. Originally developed in the 1980's; it took years for 3D printers to be commercially viable and time efficient. \#3dpers have many uses outside of printing models or doing mock up designs; they also have uses in medical fields and in true manufacturing. With the increasing growth in desktop manufacturing, being able to use complex machines in a desktop environment, has greatly helped the usability and affordability of modern \#3dpers. 

This method of printing is called fused deposition modeling, or FDM. FDM printing uses molten material layered on top of each other to produce three dimensional objects. This allows the printer to be cheap and relatively accurate. There are other methods such as: Digital Light Processing(DLP), Stereolithography(SLA), Selective Laser Sintering (SLS), Selective laser melting (SLM), Electronic Beam Melting (EBM), and Laminated object manufacturing (LOM). FDM is the most popular method for commercialized \#3dping because it does not require many specialized components to operate; just four stepper motors and an extruder a basic \#3dper. FDM printing is also more viable for businesses to mass produce test parts because they can afford the more expensive printers that print in stronger materials for testing parts. SLA and DLP are the two forms of laser printing. They both have a reservoir full of a photo-polymer resin, but after that they start to differ. SLA printers use ultraviolet light as the resin hardener, while DLP printers typically use an arc lamp and pass that light through a liquid crystal bed to print the object. These types of printers produce exceptional quality but are very costly since they both need high powered lasers to operate correctly. These types of printers work by filling the reservoir with resin, then placing the print head in the resin. Then the print head starts to slowly pull out of the resin and as it pulls out it flashes the laser in the specific patterns to produce the object. This gives the effect of making the print head look like it is dragging the object out of the resin. The next category of printers is the powder based printers, which include SLS, SLM, and EBM printers. These printers use a powder based material and slowly harden the material to form the desired object. This is done with a two piston system one acts as a reservoir for the powder and the other the print bed. The reservoir is filled with powder and then a roller slowly pushes more powder over to the print bed, as the printer hardens the layer the pistons move opposite of each other, the bed moving down and the reservoir moving up, until the print is finished. The differences between these printers is that SLS uses mostly plastic based powders, mainly nylon or other similar plastics, SLM uses metal powders, and EBM also uses metal powders but instead of using a laser based print head it uses an electron beam to do the hardening. The powder based printers have tremendous print stability, since they require no support material as the part is always submerged in powder acting as the the support material. The final type of printer is the LOM printer, it uses sheets of paper coated with adhesive that are cut layer by layer and heated into place. This method of printing is relatively cheap, not counting wasted material, and is also fast; making it a good beginners choice to \#3dping.

SLA printing is also ``the oldest [method] in [the] history of 3D printing it’s still being used nowadays'' (``Types of 3D Printers''). FDM printing was patented by Charles Hull in 1986. Charles Hull figured out that he could take layers of resin and cure them with a uv laser to create a three dimensional object. Although this was the first modern implementation of \#3dping, it was not the first to use the same method of creating a three dimensional object. \#3dping can trace it's roots back to 1860. A photographer, Fran\c cois Will\'eme, was one of the first documented cases of using composite photographs to try and make a three dimensional object. He placed 24 cameras equally around a room and took photos of a person and used the photos to try and make a \$3d model of his subjects. There were some other attempts in the one hundred and twenty six years between the two inventors, but the true birth of the modern method of \#3dping was with Charles Hull. But Hull was not the onyl inventor to try and work on \#3dping. S. Scott Scrump was also working on his method of \#3dping; this method involved using molten plastic laid in layers on top of eachother to form the \$3d model. This was the founding of the most common form today, the FDM printer. Hull knew that \#3dping would take many years to be commercially viable, but in that time \#3dping has taken major strides. Such as printing a replacement jaw for a women with chronic infections, printing a replacement kidney that can actually filter urine, or lastly a robotic aircraft with elliptical wings to give it better aerodynamics.

The basic steps that a \#3dper takes is a model must be designed or found, the model must be turned into g-code, and then the g-code must be sent to the printer. Most models are designed with CAD, computer aided design, software or other 3d modeling software. The major difference between a CAD software or a modeling software is that CAD is more designed around exact measurements and designing working prototypes. Whereas a modeling software is more designed around artistic uses. Both softwares can be used almost interchangeably but they have their own pros and cons. A better example of a CAD software is Fusion 360, one of the most popular beginner programs. It allows you to build constrained objects and easily producible patterns; an example is you need to build a replica of tesla model-s but it needs to be exact in every curve, even down to the screws. Fusion 360 lets you build individual components then assemble them all into a moving assembly of the model. Modeling software, such as blender, does not focus on exact units as much as it lets you mold the \$3d model like one would model clay. After the model is made it then needs to be turned into g-code so the printer can be told how to move. This is done with a specialized software, called a slicer, this software transforms the file into a series of layers which are then transformed into a series of movements to to tell the stepper motors how and when to move. The stepper motors move in predictable ``steps'', portions of a turn, so a one hundred step motor would move one hundred times before it reached its original starting position; because of this, the stepper motor can be finely controlled. Most printers have a motor for the x and y-axes and two for the z-axes, though these move at the same rate effectively acting as one stepper, so the printer needs x, y, and z coordinates for each part of the model. But it also needs a code for feed rate as well. Generally g-code follows these rules
\begin{displayquote}
``N\#\# G\#\# X\#\# Y\#\# Z\#\# F\#\# S\#\# T\#\# M\#\#

N: Line number; G: Motion; X: Horizontal position; Y: Vertical position; Z: Depth; F: Feed rate; S: Spindle speed; T: Tool selection; M: Miscellaneous functions; I and J: Incremental center of an arc; R: Radius of an arc'' (``Autodesk'') 
\end{displayquote}
Although these rules are generally more for CNC, computer numerical control, machines; they still are generally the same for most printers. The line number is the specific line number; there may be multiple lines per level. the motion is the type of motion, such as linear or curved. X, Y, and Z correspond to the width, length, and height in that order. Next you would likely change the feed rate, how fast the material is fed into the hot end, the spindle speed is for CNC machines. The numbers after the letter determine what type of function is used or how many steps are used. After all of this it is then sent to the printer where it is then printed; assuming nothing goes wrong.

There are many things that can go wrong with a \#3dper, there are 4 stepper motors, there is the material feed motor, there is the extruder or hot-end, the level-ness of the bed, and there might be a heated bed. All of these parameters are assuming a traditional FDM printer setup. If the bed is not heated and you have not placed any adhesive to the bed, then you can create ``spaghetti'', which is the printed plastic becoming stuck to the nozzle and dragging the mess around the bed. If you have your x and or y axes misaligned then you can create bulging prints in one direction or the other. When the z motors are misaligned then you get leaning prints, or you can nudge the bed with the print head and create more spaghetti. If the hot-end is too hot it can cause the extruded plastic to melt the previously printed layers, making the print look like it is melting. Conversely if the hot-end is too cold, the plastic will look like it has beads forming at the start of the lines, or in extreme cases cause a jam and broken plastic in the feeding tube since it has not melted at all. Sometimes the feeding motor can push too much plastic into the nozzle also causing beads to form on the print; the feeding motor can also push too little plastic which will cause the print to look like it is missing pieces of itself. The feeding motor can also have retraction errors; if the motor does not retract enough it can cause lines to join parts that are supposed to be left open, like a window. If the motor retracts too much then it can cause pieces of the print to be missing, as the motor is feeding plastic that is not there. \#3dpers can have many issues causing print failures. But with trial and error; testing each potential problem one at a time; can help solve many print issues. The problem with testing a printer is it can take many hours or even days, as printing a small object can take anywhere from an hr to multiple hours. All of these problems are hardware problems but what if you have software problems. If your slicer is does not have the correct dimensions for the print bed, it can cause wasted plastic to pour over the sides of the print bed. If the slicer is not interpreting the model correctly than it will tell the \#3dper the wrong g-code coordinates. The mechanisms that make modern desktop manufacturing possible are very complex, but still very prone to user error.

3D printing has many uses beyond hobbyist projects, and quick prototyping. It can also be used in many other fields, such as the medical field. 3D printing is a great way to test new prosthetics or even implants. With the advancement of materials and print quality, surgeons have printed braces and other implants for complicated surgeries. Some Medical professionals are starting to \#3dp scaffolds for use in surgeries. These 
\begin{displayquote}
``Scaffolds are three dimensional... biocompatible structures which can mimic the ECM properties (such as mechanical support, cellular activity and protein production through biochemical and mechanical interactions), and provide a template for cell attachment and stimulate bone tissue formation'' (``Bose'')
\end{displayquote}
Human bones are known for their ability to regenerate after serious injury but after some serious trauma bone grafts are necessary. There are different types of bone grafts; one type is the auto-graft which is a bone graft from somewhere else on the person's body. An Allo-graft is a graft from a deceased person's body. Any Medical procedure is laden with risk, and one of the most common in surgeries is biological rejection. If an implant is incompatible with the new host, the body will do everything in it's power to try and get rid of the implant. If the implant is able to be broken down by the body it will basically start the original problem all over again, while if the implant is non-biodegradable then it will cause an even worse infection in the hosts body. The solution to this problem is the bone scaffolds mentioned above, they are a porous so they body doesn't view them as too much of an intrusion and they are biodegradable so when they are no longer needed the body will dispose of them. The porous nature of the scaffolds also allow valuable nutrients to pass along to the inner bone that is in need of the nutrients. Medical \#3dping can also be used to print entirely new organs, although is still a few years off. But scientists have been able to print a fully working replacement kidney. This is possible because of the advances in printable material, regular PLA that most hobbyists use is not medically safe to use, it is also too rigid, whereas medical grade plastic is able to resist bacteria better and it is able to have a flex to it so the part is not so rigid. 

\begin{workscited}

\bibent
Schubert, Carl, et al. \textit{Innovations in 3D printing: a 3D overview from optics to organs.} British Journal of Ophthalmology, BMJ Publishing Group Ltd, 1 Feb. 2014, bjo.bmj.com/content/98/2/159.short\#article-bottom.

\bibent
Evans, Brian. \textit{Practical 3D Printers: the Science and Art of 3D Printing.} APress, 2012.

\bibent
Canessa, Enrique, et al. \textit{Low-Cost 3D Printing for Science, Education \& Sustainable Development.} ICTP—The Abdus Salam International Centre for Theoretical Physics, 2013.

\bibent
Pearce, Joshua M. \textit{Building Research Equipment with Free, Open-Source Hardware.} Science, 2012, pp. 1303–1304.

\bibent
Gebhardt, Andreas. \textit{Understanding additive manufacturing: Rapid Prototyping - Rapid Tooling - Rapid Manufacturing.} Hanser, 2012.

\bibent
``Types of 3D Printers or 3D Printing Technologies Overview.'' \textit{3D Printing from Scratch}, 2 Feb. 2016, 3dprintingfromscratch.com/common/types-of-3d-printers-or-3d-printing-technologies-overview/.

\bibent
\textit{Makerspace}, blogs.lawrence.edu/makerspace/history/.

\bibent
``Getting Started with G-Code | CNC Programming | Autodesk.'' \textit{Autodesk 2D and 3D Design and Engineering Software}, www.autodesk.com/industry/manufacturing/resources/manufacturing-engineer/g-code.

\bibent
Bose, Susmita, et al. ``Bone Tissue Engineering Using 3D Printing.'' \textit{Materials Today}, 2013, pp. 496–504.

\end{workscited}

\end{flushleft}
\end{document}
